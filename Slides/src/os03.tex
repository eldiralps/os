%%%%%%%%%%%%%%%%%%%%%%%%%%%%%%%%%%%%%%%%%%%%%%%%%%%%%%%%%%%%%%%%%%%%%%%%
% Beamer Presentation - LaTeX - Template Version 1.0 (10/11/12)
% This template has been downloaded from: http://www.LaTeXTemplates.com
% License: % CC BY-NC-SA 3.0 (http://creativecommons.org/)
% Modified by Rahmat M. Samik-Ibrahim

% REV386: Thu 28 Jul 2022 10:00
% REV371: Mon 28 Feb 2022 10:00
% REV370: Sun 20 Feb 2022 10:00
% REV363: Mon 20 Dec 2021 15:00
% STARTX: Wed 14 Sep 2016 10:00
%%%%%%%%%%%%%%%%%%%%%%%%%%%%%%%%%%%%%%%%%%%%%%%%%%%%%%%%%%%%%%%%%%%%%%%%%

% PACKAGES AND THEMES 
\documentclass[aspectratio=169, xcolor=table, notheorems, hyperref={pdfpagelabels=false}]{beamer}
%%%%%%%%%%%%%%%%%%%%%%%%%%%%%%%%%%%%%%%%%%%%%%%%%%%%%%%%%%%%%%%%%%%%%%%%
% Beamer Presentation - LaTeX - Template Version 1.0 (10/11/12)
% This template has been downloaded from: http://www.LaTeXTemplates.com
% License: % CC BY-NC-SA 3.0 (http://creativecommons.org/)
% Modified by Rahmat M. Samik-Ibrahim
% REV316 Wed 14 Jul 2021 13:42:41 WIB
% REV217 Tue Feb  4 15:10:30 WIB 2020
% REV198 Wed Mar 13 16:39:02 WIB 2019
% REV006 Mon Jan 22 19:10:41 WIB 2018
% REV005 Mon Oct  2 14:45:07 WIB 2017
% START  Thu Aug 25 14:15:19 WIB 2016
%%%%%%%%%%%%%%%%%%%%%%%%%%%%%%%%%%%%%%%%%%%%%%%%%%%%%%%%%%%%%%%%%%%%%%%%%

%% ZCZC NNNN
\newtheorem{example}{Example}

%%%%%%%%%%%%%%%%%%%%%%%%%%%%%%%%%%%%%%%%%%%%%%%%%%%%%%%%%%%%%%%%%%%%%%%%%

\let\Tiny=\tiny
\mode<presentation> {
% The Beamer class comes with a number of default slide themes
% which change the colors and layouts of slides. Below this is a list
% of all the themes, uncomment each in turn to see what they look like.
%\usetheme{Boadilla}
\usetheme{Madrid}
% ZCZC %%%%%%%%%%%%%%%%%%%%%%%%%%%%%%%%%%%%%%%%%%%%%%%%%%%%%%%%%%%%%%%%%%
% \usetheme{default} \usetheme{AnnArbor} \usetheme{Antibes} \usetheme{Bergen}
% \usetheme{Berkeley} \usetheme{Berlin} \usetheme{CambridgeUS} 
% \usetheme{Copenhagen} \usetheme{Darmstadt} \usetheme{Dresden}
% \usetheme{Frankfurt} \usetheme{Goettingen} \usetheme{Hannover}
% \usetheme{Ilmenau} \usetheme{JuanLesPins} \usetheme{Luebeck}
% \usetheme{Malmoe} \usetheme{Marburg} \usetheme{Montpellier}
% \usetheme{PaloAlto} \usetheme{Pittsburgh} \usetheme{Rochester}
% \usetheme{Singapore} \usetheme{Szeged} \usetheme{Warsaw}
% NNNN %%%%%%%%%%%%%%%%%%%%%%%%%%%%%%%%%%%%%%%%%%%%%%%%%%%%%%%%%%%%%%%%%%
% As well as themes, the Beamer class has a number of color themes
% for any slide theme. Uncomment each of these in turn to see how it
% changes the colors of your current slide theme.
%\usecolortheme{orchid}
%\usecolortheme{rose}
%\usecolortheme{seagull}
%\usecolortheme{seahorse}
\usecolortheme{whale}
% ZCZC %%%%%%%%%%%%%%%%%%%%%%%%%%%%%%%%%%%%%%%%%%%%%%%%%%%%%%%%%%%%%%%%%%
%\usecolortheme{albatross} \usecolortheme{beaver} \usecolortheme{beetle}
%\usecolortheme{crane} \usecolortheme{dolphin} \usecolortheme{dove}
%\usecolortheme{fly} \usecolortheme{lily} \usecolortheme{wolverine}
% NNNN %%%%%%%%%%%%%%%%%%%%%%%%%%%%%%%%%%%%%%%%%%%%%%%%%%%%%%%%%%%%%%%%%%
% To remove the footer line in all slides uncomment this line
%\setbeamertemplate{footline} 
% To replace the footer line in all slides uncomment this line
%\setbeamertemplate{footline}[page number] 
% To remove the navigation symbols from the bottom uncomment this line
\setbeamertemplate{navigation symbols}{} 
}

\usepackage{array}       % ZCZC
\usepackage{amssymb}     % ZCZC
\usepackage{bold-extra}  % ZCZC
\usepackage{booktabs}    % Allows \toprule, \midrule and \bottomrule in tables
\usepackage{caption}
\usepackage[T1]{fontenc} % ZCZC << >>
\usepackage{graphicx}    % Allows including images
\usepackage{listings}    % listing
\usepackage{lmodern}     % ZCZC
\usepackage{perpage}     % reset footnote per page
\usepackage{geometry}    % ZCZC
\usepackage{adjustbox}   % ZCZC
\usepackage{multirow}    % ZCZC

% \definecolor{links}{HTML}{2A1B81}
\definecolor{links}{HTML}{0011FF}
\hypersetup{colorlinks,linkcolor=,urlcolor=links}

% \usepackage{xcolor}
% \usepackage[colorlinks = true,
%             linkcolor = blue,
%             urlcolor  = blue,
%             citecolor = blue,
%             anchorcolor = blue]{hyperref}

\captionsetup[table]{name=Tabel}
\makeatletter
\def\input@path{{src/}}
\makeatother
\graphicspath{{src/}}      % src directory
\MakePerPage{footnote}     % reset page

% NNNN %%%%%%%%%%%%%%%%%%%%%%%%%%%%%%%%%%%%%%%%%%%%%%%%%%%%%%%%%%%%%%%%%%

%% % XXXXXXXXXXXXXXXXXXXXXXXXXXXXXXXXXXXXXXXXXXXXXXXXXXXXXXXXXXXXXXXXXXXXXXXXXX
%% % The short title appears at the bottom of every slide, 
%% % the full title is only on the title page
%% \title[Judul Pendek]{Judul Panjang dan Lengkap} 
%% \author{Cecak bin Kadal}
%% \institute[UILA]
%% {
%% University of Indonesia at Lenteng Agung \\ 
%% \medskip
%% \textit{cecak@binKadal.com}
%% }
%% \date{REV00 24-Aug-2016}
%% % \date{\today}
%% 

%% % XXXXXXXXXXXXXXXXXXXXXXXXXXXXXXXXXXXXXXXXXXXXXXXXXXXXXXXXXXXXXXXXXXXXXXXXXX
%% \begin{document}
%% \section{Judul}
%% \begin{frame}
%% \titlepage
%% \end{frame}
%% 
%% % XXXXXXXXXXXXXXXXXXXXXXXXXXXXXXXXXXXXXXXXXXXXXXXXXXXXXXXXXXXXXXXXXXXXXXXXXX
%% \section{Agenda}
%% \begin{frame}
%% \frametitle{Agenda}
%% % Throughout your presentation, if you choose to use \section{} and 
%% % \subsection{} commands, these will automatically be printed on 
%% % this slide as an overview of your presentation
%% \tableofcontents 
%% \end{frame}
%% 
%% % XXXXXXXXXXXXXXXXXXXXXXXXXXXXXXXXXXXXXXXXXXXXXXXXXXXXXXXXXXXXXXXXXXXXXXXXXX
%% \section{UUD dan Pancasila}
%% \subsection{UUD}
%% \begin{frame}
%% \frametitle{Pembukaan}
%% Bahwa sesungguhnya kemerdekaan itu ialah hak segala bangsa dan oleh 
%% sebab itu, maka penjajahan diatas dunia harus dihapuskan karena 
%% tidak sesuai dengan perikemanusiaan dan perikeadilan.
%% \\~\\
%% Atas berkat rahmat Allah Yang Maha Kuasa dan dengan didorongkan oleh 
%% keinginan luhur, supaya berkehidupan kebangsaan yang bebas, maka 
%% rakyat Indonesia menyatakan dengan ini kemerdekaannya.
%% \end{frame}
%% 
%% % XXXXXXXXXXXXXXXXXXXXXXXXXXXXXXXXXXXXXXXXXXXXXXXXXXXXXXXXXXXXXXXXXXXXXXXXXX
%% \begin{frame}
%% \frametitle{Alenia Ketiga}
%% Kemudian daripada itu untuk membentuk suatu pemerintah negara Indonesia 
%% yang melindungi segenap bangsa Indonesia dan seluruh tumpah darah Indonesia 
%% dan untuk memajukan kesejahteraan umum, mencerdaskan kehidupan bangsa, dan 
%% ikut melaksanakan ketertiban dunia yang berdasarkan kemerdekaan, perdamaian 
%% abadi dan keadilan sosial, maka disusunlah kemerdekaan kebangsaan Indonesia 
%% itu dalam suatu Undang-Undang Dasar negara Indonesia, yang terbentuk dalam 
%% suatu susunan negara Republik Indonesia yang berkedaulatan rakyat dengan 
%% berdasar kepada:
%% \begin{itemize}
%% \item Ketuhanan Yang Maha Esa,
%% \item kemanusiaan yang adil dan beradab,
%% \item persatuan Indonesia,
%% \item dan kerakyatan yang dipimpin oleh hikmat kebijaksanaan 
%%       dalam permusyawaratan/ perwakilan,
%% \item serta dengan mewujudkan suatu keadilan sosial bagi seluruh rakyat 
%%       Indonesia.
%% \end{itemize}
%% \end{frame}
%% 
%% % XXXXXXXXXXXXXXXXXXXXXXXXXXXXXXXXXXXXXXXXXXXXXXXXXXXXXXXXXXXXXXXXXXXXXXXXXX
%% \subsection{Pancasila}
%% \begin{frame}
%% \frametitle{Tujuh Kunci Pokok}
%% \begin{block}{Pertama - Kedua - Ketiga}
%% Indonesia ialah negara berdasarkan hukum.
%% Sistem konstitusional.
%% Kekuasaan negara tertinggi di tangan MPR.
%% \end{block}
%% 
%% \begin{block}{Keempat - Kelima}
%% Presiden adalah penyelenggara pemerintahan tertinggi di bawah MPR.
%% Adanya pengawasan DPR.
%% \end{block}
%% 
%% \begin{block}{Keenam}
%% Menteri negara adalah pembantu presiden dan tidak bertanggung jawab 
%% kepada DPR.
%% \end{block}
%% 
%% \begin{block}{Ketujuh}
%% Kekuasaan kepala negara tidak tak tebatas.
%% \end{block}
%% 
%% \end{frame}
%% 
%% % XXXXXXXXXXXXXXXXXXXXXXXXXXXXXXXXXXXXXXXXXXXXXXXXXXXXXXXXXXXXXXXXXXXXXXXXXX
%% \section{Rupa-rupa}
%% \subsection{Kolom}
%% \begin{frame}
%% \frametitle{Kolom}
%% % The "c" option specifies centered vertical alignment 
%% % while the "t" option is used for top vertical alignment
%% \begin{columns}[c] 
%% % Left column and width
%% \column{.45\textwidth} 
%% \textbf{Heading}
%% \begin{enumerate}
%% \item Satu-satu
%% \item Dua-dua
%% \item Tiga-tiga
%% \item Satu-dua-tiga
%% \end{enumerate}
%% 
%% % Right column and width
%% \column{.5\textwidth}
%% Satu-satu~\dots{} aku sayang ibu!
%% Dua-dua~\ldots{} juga sayang ayah!
%% Tiga-tiga~\ldots{} sayang adik kakak!
%% Satu-dua-tiga~\ldots{} sayang semuanya!
%% 
%% \end{columns}
%% \end{frame}
%% 
%% % XXXXXXXXXXXXXXXXXXXXXXXXXXXXXXXXXXXXXXXXXXXXXXXXXXXXXXXXXXXXXXXXXXXXXXXXXX
%% \subsection{Tabel}
%% \begin{frame}
%% \frametitle{Tabel}
%% \begin{table}
%% \begin{tabular}{l l l}
%% \toprule
%% \textbf{Nama} & \textbf{NPM} & \textbf{Tanggal Lahir}\\
%% \midrule
%% Cecak bin Kadal & 1234567890 & 1 Jan 2015 \\
%% Aneh bin Ajaib  & 0987654321 & 31 Des 2014 \\
%% \bottomrule
%% \end{tabular}
%% \caption{Keterangan Tabel}
%% \end{table}
%% \end{frame}
%% 
%% % XXXXXXXXXXXXXXXXXXXXXXXXXXXXXXXXXXXXXXXXXXXXXXXXXXXXXXXXXXXXXXXXXXXXXXXXXX
%% \subsection{Teori}
%% \begin{frame}
%% \frametitle{Teori}
%% \begin{theorem}[Teori Satu Batu]
%% $E = mc^2$
%% \end{theorem}
%% \end{frame}
%% 
%% % XXXXXXXXXXXXXXXXXXXXXXXXXXXXXXXXXXXXXXXXXXXXXXXXXXXXXXXXXXXXXXXXXXXXXXXXXX
%% \subsection{Verbatim}
%% % Need to use the fragile option when verbatim is used in the slide
%% \begin{frame}[fragile] 
%% \frametitle{Verbatim}
%% \begin{example}[Teori Satu Batu]
%% \begin{verbatim}
%% \begin{theorem}[Teori Satu Batu]
%% $E = mc^2$
%% \end{theorem}
%% \end{verbatim}
%% \end{example}
%% \end{frame}
%% 
%% % XXXXXXXXXXXXXXXXXXXXXXXXXXXXXXXXXXXXXXXXXXXXXXXXXXXXXXXXXXXXXXXXXXXXXXXXXX
%% \subsection{Gambar}
%% \begin{frame}
%% \frametitle{Gambar}
%% \begin{figure}
%% \includegraphics[width=0.5\linewidth]{2}
%% \caption{Ini Gambar JPG}
%% \end{figure}
%% \end{frame}
%% 
%% % XXXXXXXXXXXXXXXXXXXXXXXXXXXXXXXXXXXXXXXXXXXXXXXXXXXXXXXXXXXXXXXXXXXXXXXXXX
%% \subsection{Rujukan}
%% % Need to use the fragile option when verbatim is used in the slide
%% \begin{frame}[fragile] 
%% \frametitle{Rujukan dan Kutipan}
%% Contoh penggunaan \verb|\cite| ketika mengutip\cite{p1}.
%% Perhatian: Beamer tidak mengerti \verb|\BibTeX|~\ldots
%% \footnotesize{
%%   \begin{thebibliography}{99} 
%%   \bibitem[Smith, 2012]{p1} John Smith (2012)
%%      \newblock Katak dalam Tempurung
%%      \newblock \emph{Jurnal Kelapa dan Amfibi} 12(3), 45 -- 678.
%%   \end{thebibliography}
%% }
%% \end{frame}
%% 
%% % XXXXXXXXXXXXXXXXXXXXXXXXXXXXXXXXXXXXXXXXXXXXXXXXXXXXXXXXXXXXXXXXXXXXXXXXXX
%% \subsection{Selesai}
%% \begin{frame}
%% \Huge{\centerline{Selesai}}
%% \end{frame}
%% 
%% % XXXXXXXXXXXXXXXXXXXXXXXXXXXXXXXXXXXXXXXXXXXXXXXXXXXXXXXXXXXXXXXXXXXXXXXXXX
%% \end{document}

\newcommand{\revision}{REV392 30-Aug-2022}
% w! tmptmp
% REV392: Tue 30 Aug 2022 12:00
% REV389: Mon 15 Aug 2022 08:00
% REV379: Tue 17 May 2022 05:00
% REV359: Sat 30 Oct 2021 14:00
% REV339: Sat 04 Sep 2021 12:00
% STARTS: Wed 24 Aug 2016 19:00
%%%%%%%%%%%%%%%%%%%%%%%%%%%%%%%%%%%%%
\newcommand{\kopikopi}{\textcopyright{}2016-2022 CBKadal + VauLSMorg}



% XXXXXXXXXXXXXXXXXXXXXXXXXXXXXXXXXXXXXXXXXXXXXXXXXXXXXXXXXXXXXXXXXXXXXXXXXX
% The short title appears at the bottom of every slide, 
% the full title is only on the title page
% \date{\today}
\title[\kopikopi]
{CSGE602055 Operating Systems \\ 
CSF2600505 Sistem Operasi \\
Week 03:
File System \& FUSE}
\author{C. BinKadal}
\institute[SDN]
{
Sendirian Berhad\\
\medskip
\url{https://os.vlsm.org/Slides/os03.pdf}
\\ \texttt{Always check for the latest revision!}
}
\date{\revision}

% XXXXXXXXXXXXXXXXXXXXXXXXXXXXXXXXXXXXXXXXXXXXXXXXXXXXXXXXXXXXXXXXXXXXXXXXXX
\begin{document}

\lstset{
basicstyle=\ttfamily\tiny, % \tiny \small \footnotesize 
breakatwhitespace=true,
language=C,
columns=fullflexible,
keepspaces=true,
breaklines=true,
tabsize=3, 
showstringspaces=false,
extendedchars=true}

\section{Start}
\begin{frame}
\titlepage
\end{frame}

% XXXXXXXXXXXXXXXXXXXXXXXXXXXXXXXXXXXXXXXXXXXXXXXXXXXXXXXXXXXXXXXXXXXXXXXXXX

%%%%%%%%%%%%%%%%%%%%%%%%%%%%%%%%%%%%%%%%%%%%%%%%%%%%%%%%%%%%%%%%%%%%%%%%%
% REV352 Sun 10 Oct 2021 09:56:47 WIB
% REV341 Sun 05 Sep 2021 23:30:00 WIB
% REV333 Thu 26 Aug 2021 08:52:24 WIB
% REV328 Sat 14 Aug 2021 06:32:08 WIB
% REV272 Mon 01 Mar 2021 12:02:09 WIB
% START0 Sat Sep  2 10:51:33 WIB 2017
%%%%%%%%%%%%%%%%%%%%%%%%%%%%%%%%%%%%%%%%%%%%%%%%%%%%%%%%%%%%%%%%%%%%%%%%%

\begin{frame}[fragile]
\section{Schedule}
\frametitle{OS212\footnote{%
) This information will be on \textbf{EVERY} page two (2) of this course material.}): 
Operating Systems 2021 - 2}
\scalebox{0.73}{%
\begin{tabular}{|c|c|c|c|}
\hline
\makebox[106pt]{OS A} & \makebox[106pt]{OS B} & \makebox[107pt]{OS C} & \makebox[107pt]{OS INT} \\
\hline
\multicolumn{4}{|c|}{Every first day of the Week, \textbf{Quiz\#1:} (07:40-07:50) and \textbf{Quiz\#2:} 07:20-07:40} \\
\hline
Monday/Thursday & Monday/Thursday & Monday/Thursday & Monday/Wednesday   \\
13:00 --- 14:40  & 15:00 --- 16:40\footnote{) \textbf{OS B:} Week00-Week05 (RMS); Week06-Week10 (MAM).} &
                                      13:00 --- 14:40 & 08:00 --- 09:40  \\
14:00 --- finish & 16:00 --- finish & 13:00 --- 14:40 & 09:00 --- finish \\
\hline
\end{tabular}
}

\vspace{5pt}

\scalebox{0.73}{%
\begin{tabular}{|c|c|l|l|}
\hline
\textbf{Week} & \textbf{Schedule \& Deadline}\footnote{%
    ) The \textbf{DEADLINE} of Week 00 is 05 Sep 2021,
      whereas the \textbf{DEADLINE} of Week 01 is 12 Sep 2021, and so on...%
    })& \textbf{Topic} & \textbf{OSC10}\footnote{%
    ) Silberschatz et. al.: \textbf{Operating System Concepts}, $10^{th}$ Edition, 2018.}) \\
\hline
Week 00  & 30 Aug - 05 Sep 2021 & Overview 1, Virtualization \& Scripting & Ch. 1, 2, 18. \\
Week 01  & 06 Sep - 12 Sep 2021 & Overview 2, Virtualization \& Scripting & Ch. 1, 2, 18. \\
Week 02  & 13 Sep - 19 Sep 2021 & Security, Protection, Privacy, \& C-language.  & Ch. 16, 17. \\
Week 03  & 20 Sep - 26 Sep 2021 & File System \& FUSE  & Ch. 13, 14, 15. \\
Week 04  & 27 Sep - 03 Oct 2021 & Addressing, Shared Lib, \& Pointer & Ch. 9. \\
Week 05  & 04 Oct - 10 Oct 2021 & Virtual Memory & Ch. 10. \\
\hline
Week 06  & 11 Oct - 31 Oct 2021 & Concurrency: Processes \& Threads & Ch. 3, 4. \\
Week 07  & 01 Nov - 07 Nov 2021 & Synchronization \& Deadlock & Ch. 6, 7, 8. \\
Week 08  & 08 Nov - 14 Nov 2021 & Scheduling + W06/W07 & Ch. 5. \\
Week 09  & 15 Nov - 21 Nov 2021 & Storage, Firmware, Bootloader, \& Systemd & Ch. 11. \\
Week 10  & 22 Nov - 28 Nov 2021 & I/O \& Programming & Ch. 12. \\%
% MidTerm  & 00 XXX 2020 (XX:XX-XX:XX) & MidTerm (UTS) & \cellcolor{red!44} TBA! \\
% Reserved & 00 XXX - 00 XXX 2020 & Q \& A & \\
% Final    & 00 XXX 2020 XX:XX & First Part Final  (UAS tahap I)  & \cellcolor{red!44} This schedule is   \\
% Extra    & NA & No Extra assignment & \cellcolor{red!44} subject to change. \\
\hline
\end{tabular}
}
\end{frame}

\begin{frame}[fragile]
\frametitle{\textbf{STARTING POINT} --- 
{
\definecolor{links}{HTML}{FDEE00}
\hypersetup{colorlinks,linkcolor=,urlcolor=links}
\url{https://os.vlsm.org/}
}
}
\begin{itemize}
\item[$\square$] \textbf{Text Book} ---
                 Any recent/decent OS book. Eg. (\textbf{OSC10}) Silberschatz et. al.: 
                 \textbf{Operating System Concepts}, $10^{th}$ Edition, 2018.
                 See also \url{https://www.os-book.com/OS10/}.
\item[$\square$] \textbf{Resources}
\begin{itemize}
\item[$\square$] \href{https://scele.cs.ui.ac.id/course/view.php?id=3268}{\textbf{SCELE OS212}} ---
\url{https://scele.cs.ui.ac.id/course/view.php?id=3268}.\\
The enrollment key is \textbf{XXX}.
\item[$\square$] \textbf{Download Slides and Demos from GitHub.com} \\
\url{https://github.com/UI-FASILKOM-OS/SistemOperasi/}:

                 {\scriptsize%
                 \href{https://os.vlsm.org/Slides/os00.pdf}{\texttt{os00.pdf} (W00)},
                 \href{https://os.vlsm.org/Slides/os01.pdf}{\texttt{os01.pdf} (W01)},
                 \href{https://os.vlsm.org/Slides/os02.pdf}{\texttt{os02.pdf} (W02)},
                 \href{https://os.vlsm.org/Slides/os03.pdf}{\texttt{os03.pdf} (W03)},

                 \href{https://os.vlsm.org/Slides/os04.pdf}{\texttt{os04.pdf} (W04)},
                 \href{https://os.vlsm.org/Slides/os05.pdf}{\texttt{os05.pdf} (W05)},
                 \href{https://os.vlsm.org/Slides/os06.pdf}{\texttt{os06.pdf} (W06)},
                 \href{https://os.vlsm.org/Slides/os07.pdf}{\texttt{os07.pdf} (W07)},

                 \href{https://os.vlsm.org/Slides/os08.pdf}{\texttt{os08.pdf} (W08)},
                 \href{https://os.vlsm.org/Slides/os09.pdf}{\texttt{os09.pdf} (W09)},
                 \href{https://os.vlsm.org/Slides/os10.pdf}{\texttt{os10.pdf} (W10)}.
                 }
\item[$\square$] \textbf{Problems}\\
                 {\scriptsize% 
                 \href{https://rms46.vlsm.org/2/195.pdf}{\texttt{195.pdf} (W00)},
                 \href{https://rms46.vlsm.org/2/196.pdf}{\texttt{196.pdf} (W01)},
                 \href{https://rms46.vlsm.org/2/197.pdf}{\texttt{197.pdf} (W02)},
                 \href{https://rms46.vlsm.org/2/198.pdf}{\texttt{198.pdf} (W03)},\\
                 \href{https://rms46.vlsm.org/2/199.pdf}{\texttt{199.pdf} (W04)},
                 \href{https://rms46.vlsm.org/2/200.pdf}{\texttt{200.pdf} (W05)},
                 \href{https://rms46.vlsm.org/2/201.pdf}{\texttt{201.pdf} (W06)},
                 \href{https://rms46.vlsm.org/2/202.pdf}{\texttt{202.pdf} (W07)},\\
                 \href{https://rms46.vlsm.org/2/203.pdf}{\texttt{203.pdf} (W08)},
                 \href{https://rms46.vlsm.org/2/204.pdf}{\texttt{204.pdf} (W09)},
                 \href{https://rms46.vlsm.org/2/205.pdf}{\texttt{205.pdf} (W10)}.}
\item[$\square$] \textbf{LFS} --- \url{http://www.linuxfromscratch.org/lfs/view/stable/}
\item[$\square$] \textbf{OSP4DISS} --- \url{https://osp4diss.vlsm.org/}
\item[$\square$] \textbf{DOIT} --- \url{https://doit.vlsm.org/001.html}
\end{itemize}
\end{itemize}
\end{frame}



% XXXXXXXXXXXXXXXXXXXXXXXXXXXXXXXXXXXXXXXXXXXXXXXXXXXXXXXXXXXXXXXXXXXXXXXXXX
% Throughout your presentation, if you choose to use \section{} and 
% \subsection{} commands, these will automatically be printed on 
% this slide as an overview of your presentation
\section{Agenda}
\begin{frame}{Outline}
  \frametitle{Agenda}
  \tableofcontents[sections={1-}]
\end{frame}
% \begin{frame}
   % \frametitle{Agenda (2)}
   % \tableofcontents[sections={12-}]
% \end{frame}

% XXXXXXXXXXXXXXXXXXXXXXXXXXXXXXXXXXXXXXXXXXXXXXXXXXXXXXXXXXXXXXXXXXXXXXXXXX

%%%%%%%%%%%%%%%%%%%%%%%%%%%%%%%%%%%%%%%%%%%%%%%%%%%%%%%%%%%%%%%%%%%%%%%%%
% REV154 Thu Aug 23 11:22:02 WIB 2018
% START0 Thu Jul 26 20:01:45 WIB 2018
%%%%%%%%%%%%%%%%%%%%%%%%%%%%%%%%%%%%%%%%%%%%%%%%%%%%%%%%%%%%%%%%%%%%%%%%%

\section{Week 03}
\begin{frame}[fragile]
\frametitle{Week 03 File System \& FUSE:
Topics\footnote{Source: ACM IEEE CS Curricula 2013}}

\begin{itemize}
\item Files: data, metadata, operations, organization, buffering, sequential, nonsequential
\item Directories: contents and structure
\item File systems: partitioning, mount/unmount, virtual file systems
\item Standard implementation techniques
\item Memory-mapped files
\item Special-purpose file systems
\item Naming, searching, access, backups
\item Journaling and log-structured file systems
\end{itemize}
\end{frame}

\begin{frame}[fragile]
\frametitle{Week 03 File System \& FUSE:
Learning Outcomes\footnote{Source: ACM IEEE CS Curricula 2013}}
\begin{itemize}
\item Describe the choices to be made in designing file systems. [Familiarity]
\item Compare and contrast different approaches to file organization, recognizing the strengths and weaknesses of each. [Usage]
\item Summarize how hardware developments have led to changes in the priorities for the design and the management of file systems. [Familiarity]
\item Summarize the use of journaling and how log-structured file systems enhance fault tolerance. [Familiarity]
\end{itemize}

\end{frame}



% XXXXXXXXXXXXXXXXXXXXXXXXXXXXXXXXXXXXXXXXXXXXXXXXXXXXXXXXXXXXXXXXXXXXXXXXXX
\section{OSC10 (Silberschatz) Chapter 13, 14, and 15}
\begin{frame}
\frametitle{OSC10 (Silberschatz) Chapter 13: File-System Interface, Chapter 14: File System Implementation, and Chapter 15: File System Internals}
\begin{multicols}{3}
  \begin{itemize}
  \item OSC10 Chapter 13
  \begin{itemize}
  \item File Concept
  \item Access Methods
  \item Disk and Directory Structure
  \item Protection
  \item Memory-Mapped Files
  \end{itemize}
  \end{itemize}
  \vfill \null
\columnbreak
  \begin{itemize}
  \item OSC10 Chapter 14
  \begin{itemize}
  \item File-System Structure
  \item File-System Operations
  \item Directory Implementation
  \item Allocation Methods
  \item Free-Space Management
  \item Efficiency and Performance
  \item Recovery
  \item Example: WAFL File System
  \end{itemize}
  \end{itemize}
  \vfill \null
\columnbreak
  \begin{itemize}
  \item OSC10 Chapter 15
  \begin{itemize}
  \item File Systems
  \item File-System Mounting
  \item Partitions and Mounting
  \item File Sharing
  \item Virtual File Systems
  \item Remote File Systems
  \item Consistency Semantics
  \item NFS
  \end{itemize}
  \end{itemize}
  \vfill \null
\end{multicols}
\end{frame}

% XXXXXXXXXXXXXXXXXXXXXXXXXXXXXXXXXXXXXXXXXXXXXXXXXXXXXXXXXXXXXXXXXXXXXXXXXX
\section{File System Interface}
\begin{frame}[fragile]
\frametitle{File System Interface}
\begin{itemize}
\item File Concept
\begin{itemize}
\item File Attributes: Name, Id, Type, Location, Size, Protection, Time Stamp: create, last modified, last accessed.
\item File Operation
\begin{itemize}
\item Create/Delete/Truncate
\item Open/Close
\item Read/Write
\end{itemize}
\item File Types: Executable, Object, Source Code, Library, Markup, Markdown, Archive, Compressed.
\item File Structure: No Structure (just a string).
\item Access Methods: Sequential vs Direct Access
\end{itemize}
\item Directory and Disk Structure
\begin{itemize}
\item Three-Structured Directories
\item Directory Operation: create/delete, search/list, rename, traverse
\item Path Name: Absolute vs. Relative
\item FS Mounting vs. Volume Based System
\end{itemize}
\item File Sharing
\item Protection: Access Control (eg. -rwx--x--x)
\end{itemize}
\end{frame}

% XXXXXXXXXXXXXXXXXXXXXXXXXXXXXXXXXXXXXXXXXXXXXXXXXXXXXXXXXXXXXXXXXXXXXXXXXX
\section{File System Organization}
\begin{frame}[fragile]
\frametitle{File System Organization}
\begin{itemize}
\item Disk Partition
\begin{itemize}
\item One Disk --- Many Partitions
\item Many Disks --- One Partitions
\item Many Disks --- Many Partitions
\item One Partition --- One File System (Volume)
\end{itemize}
\item Mounting vs. Volumes
\end{itemize}
% \begin{lstlisting}[basicstyle=\ttfamily\tiny]
\begin{lstlisting}[basicstyle=\ttfamily\footnotesize]
demo@badak:~$ df
Filesystem     1K-blocks     Used Available Use% Mounted on
/dev/sda2        9515660  1435776   7573468  16% /
/dev/sdb1       32895760 12156672  19045036  39% /usr
/dev/sdc1      412322216 79695252 311639116  21% /home
udev               10240        0     10240   0% /dev
tmpfs           16508828        0  16508828   0% /dev/shm
tmpfs            6603532     8880   6594652   1% /run
tmpfs               5120        0      5120   0% /run/lock
tmpfs           16508828        0  16508828   0% /sys/fs/cgroup
tmpfs            3301768        0   3301768   0% /run/user/1002
demo@badak:~$ 
\end{lstlisting}
\end{frame}

% XXXXXXXXXXXXXXXXXXXXXXXXXXXXXXXXXXXXXXXXXXXXXXXXXXXXXXXXXXXXXXXXXXXXXXXXXX
\section{FHS: Filesystem Hierarchy Standard}
\begin{frame}[fragile]
\frametitle{FHS: Filesystem Hierarchy Standard}
\begin{itemize}
\item {\scriptsize Source (URL) \url{http://refspecs.linuxfoundation.org/FHS_3.0/fhs-3.0.pdf}}
\item A file placement guidelines/requirements for GNU/Linux-like OS.
\item[]
\scalebox{0.61}{%\multirow{2}{*}{XXX}   
\begin{tabular}{|l|l|l|}
\hline
FILES                                 & shareable (multiple hosts) & unshareable (single hosts) \\
\hline
static (read only, except for update) & /usr, /opt                 & /etc, /boot                \\
\hline
variable (r/w)                        & /var/mail, /var/spool/news & /var/run, /var/lock        \\
\hline
\end{tabular}
}
\item The Root File System (Required)
\item[]
\scalebox{0.61}{%
\begin{tabular}{l|l}
Directory & Description \\
\hline
/bin   & Essential command binaries \\
/boot  & Static files of the boot loader\\
/dev   & Device files\\
/etc   & Host-specific system configuration\\
/lib   & Essential shared libraries and kernel modules\\
/media & Mount point for removable media\\
/mnt   & Mount point for mounting a filesystem temporarily\\
/opt   & Add-on application software packages\\
/run   & Data relevant to running processes\\
/sbin  & Essential system binaries\\
/srv   & Data for services provided by this system\\
/tmp   & Temporary files\\
/usr   & Secondary hierarchy\\
/var   & Variable data \\
\end{tabular}
}

\end{itemize}
\end{frame}

% XXXXXXXXXXXXXXXXXXXXXXXXXXXXXXXXXXXXXXXXXXXXXXXXXXXXXXXXXXXXXXXXXXXXXXXXXX
\begin{frame}[fragile]
\frametitle{More FHS 1}
\begin{itemize}
\item  Specific Options
\item[]
\scalebox{0.69}{%
\begin{tabular}{l|l}
Directory & Description \\
\hline
/home      & User home directories (optional) \\
/lib<qual> & Alternate format essential shared libraries(optional) \\
/root       & Home directory for the root user (optional) \\
\end{tabular}
}
\item The /usr Hierarchy
\item[]
\scalebox{0.69}{%
\begin{tabular}{l|l}
Directory & Description \\
\hline
/usr/bin & Most user commands (required)\\
/usr/lib & Libraries (required)\\
/usr/local & Local hierarchy (empty after main installation) (required)\\
          & /usr/local/\{bin|etc|games|include|lib|man|sbin|share|src\} (required) \\
/usr/sbin & Non-vital system binaries (required)\\
/usr/share & Architecture-independent data (required)\\
          & /usr/share/\{man|misc\} (required)\\
          & /usr/share/\{color|dict|doc|games|info|locale\} (optional)\\
          & /usr/share/\{nls|ppd|sgml|terminfo|tmac|xml|zoneinfo\} (optional)\\
\hline
/usr/games   & Games and educational binaries (optional)\\
/usr/include & Header files included by C programs (optional)\\
/usr/libexec & Binaries run by other programs (optional)\\
/usr/lib<qual> & Alternate Format Libraries (optional)\\
/usr/src     & Source code (optional)\\
\end{tabular}
}

\end{itemize}
\end{frame}

% XXXXXXXXXXXXXXXXXXXXXXXXXXXXXXXXXXXXXXXXXXXXXXXXXXXXXXXXXXXXXXXXXXXXXXXXXX
\begin{frame}[fragile]
\frametitle{More FHS 2}
\begin{itemize}
\item The /var Hierarchy
\item[]
\scalebox{0.72}{%
\begin{tabular}{l|l}
Directory & Description \\
\hline
/var/cache & Application cache data (required)\\
/var/lib   & Variable state information (required)\\
           & /var/lib/misc (required)\\
/var/local & Variable data for /usr/local (required)\\
/var/lock  & Lock fileslogLog files and directories (required)\\
/var/opt   & Variable data for /opt (required)\\
/var/run   & Data relevant to running processes (required)\\
/var/spool & Application spool data (required)\\
/var/tmp   & Temporary files preserved between system reboots (required)\\
\hline
/var/backups  & (reserved names, do not use)\\
/var/cron     & (reserved names, do not use)\\
/var/msgs     & (reserved names, do not use)\\
/var/preserve & (reserved names, do not use)\\
\hline
/var/account  & Process accounting logs (optional)\\
/var/crash    & System crash dumps (optional)\\
/var/games    & Variable game data (optional)\\
/var/mail     & User mailbox files (optional)\\
/var/yp       & Network Information Service (NIS) database files(optional)\\
\end{tabular}
}

\end{itemize}
\end{frame}

% XXXXXXXXXXXXXXXXXXXXXXXXXXXXXXXXXXXXXXXXXXXXXXXXXXXXXXXXXXXXXXXXXXXXXXXXXX
\begin{frame}[fragile]
\frametitle{More FHS 3}
\begin{itemize}
\item (Mostly) Linux
\item[]
\scalebox{0.92}{%
\begin{tabular}{l|l}
Directory & Description \\
\hline
/proc           & Kernel and process information virtual filesystem\\
/sys            & Kernel and system information virtual filesystem\\
/usr/include    & Header files included by C programs\\
/usr/src        & Source code\\
/var/spool/cron & cron and at jobs\\
\end{tabular}
}

\end{itemize}
\end{frame}

% XXXXXXXXXXXXXXXXXXXXXXXXXXXXXXXXXXXXXXXXXXXXXXXXXXXXXXXXXXXXXXXXXXXXXXXXXX

\section{Devices}
\begin{frame}[fragile]
\frametitle{Devices}
\begin{itemize}
\item the \texttt{/dev/} directory
\begin{itemize}
\item \texttt{/etc/fstab}: configuration of filesystems
\item \texttt{/etc/mtab} $\rightarrow$ \texttt{/proc/mounts}: mounted filesystems
\item \texttt{/proc/swaps}: swap filesystems
\item \texttt{df}: checking diskspace and filesystems
\item Device Major and Minor Numbers
\item UUID - Universally Unique IDentifier (128 bits)
\item GUID - Globally Unique IDentifiers: \texttt{ls -al /dev/disk/by-uuid}
\item practically is NOT guaranteed unique
\item FUSE: Filesystem in Userspace
\item More Storage Structure
\begin{itemize}
\item \texttt{tmpfs} --- a temporary file storage, stored in RAM that grows and shrinks.
\item \texttt{objfs} --- dynamic kernel object filesystem.
\item \texttt{ctfs} --- (creating, controlling, and observing) contract file system .
\item \texttt{loopfs} --- loop filesystem allows to dynamically allocate loop devices.
\item \texttt{procfs} --- proc filesystem presents information about processes.
\item \texttt{ufs} --- the original Unix Filesystem (before Linux ext2).
\item \texttt{zfs} --- the Zettabyte Filesystem is  both a volume manager and a file system. 
\end{itemize}
\end{itemize}
\end{itemize}
\end{frame}

% XXXXXXXXXXXXXXXXXXXXXXXXXXXXXXXXXXXXXXXXXXXXXXXXXXXXXXXXXXXXXXXXXXXXXXXXXX
\begin{frame}[fragile]
\frametitle{A Typical Ubuntu 20.04 Work Station}
\begin{lstlisting}[basicstyle=\ttfamily\tiny]
rms46@pamulang1:~$ df
Filesystem      1K-blocks       Used Available Use% Mounted on
udev              8138664          0   8138664   0% /dev
tmpfs             1634140       1948   1632192   1% /run
tmpfs             8170684     210348   7960336   3% /dev/shm
tmpfs                5120          4      5116   1% /run/lock
tmpfs             8170684          0   8170684   0% /sys/fs/cgroup
tmpfs             1634136         76   1634060   1% /run/user/1000
/dev/sda1           98304      33523     64781  35% /boot/efi
/dev/sda3       286082372   78565916 207516456  28% /altfs/ntfs
/dev/sda5        32999120    9181772  22111364  30% /altfs/linux1
/dev/sda6        38186548   12054612  24162428  34% /altfs/linux2
/dev/sda7       126265680   13342928 106465768  12% /
/dev/sdb2        62216964   13238156  45788588  23% /var
/dev/sdb3      3532259904 2605226568 747535200  78% /home
/dev/loop0         101632     101632         0 100% /snap/core/10859
/dev/loop1          65920      65920         0 100% /snap/gtk-common-themes/1513
/dev/loop2          66432      66432         0 100% /snap/gtk-common-themes/1514
/dev/loop3         678016     678016         0 100% /snap/intellij-idea-community/273
/dev/loop4         679040     679040         0 100% /snap/intellij-idea-community/270
/dev/loop5          52352      52352         0 100% /snap/snap-store/498
/dev/loop6         223232     223232         0 100% /snap/gnome-3-34-1804/60
/dev/loop7         267008     267008         0 100% /snap/kde-frameworks-5-core18/32
/dev/loop8         166784     166784         0 100% /snap/gnome-3-28-1804/145
/dev/loop9         102784     102784         0 100% /snap/kotlin/57
/dev/loop10         52352      52352         0 100% /snap/snap-store/518
/dev/loop11         56832      56832         0 100% /snap/core18/1988
###########        ##### TL;DR #####         # #### #################
/dev/loop18         56832      56832         0 100% /snap/core18/1944
/dev/loop19        142080     142080         0 100% /snap/chromium/1506
\end{lstlisting}
\end{frame}

% XXXXXXXXXXXXXXXXXXXXXXXXXXXXXXXXXXXXXXXXXXXXXXXXXXXXXXXXXXXXXXXXXXXXXXXXXX
\section{File System Implementation}
\begin{frame}[fragile]
\frametitle{File Systems Implementation}
\begin{itemize}
\item File System Layers / Structure
\begin{itemize}
\item Application Programs
\item Logical File Systems
\item File-Organization Module
\item Basic File Systems
\item I/O Control
\item Hardware Device
\end{itemize}
\item File System Implementation
\item File Control Block
\item FS In Memory Structure
\item VFS: Virtual File Systems
\begin{itemize}
\item How to support multiple File Systems
\item I.e. How to support multiple \texttt{open()/close() read()/write()} operations
\end{itemize}
\end{itemize}
\end{frame}

% XXXXXXXXXXXXXXXXXXXXXXXXXXXXXXXXXXXXXXXXXXXXXXXXXXXXXXXXXXXXXXXXXXXXXXXXXX
\begin{frame}[fragile]
\frametitle{Implementation and Allocation Method}
\begin{itemize}
\item Directory Implementation
\begin{itemize}
\item Linear List
\item Hast Table
\end{itemize}
\item {Allocation Method}
\begin{itemize}
\item Contiguous
\item Linked
\item Indexed
\item Combined Scheme
\end{itemize}
\item Free Space Management
\item Performance \& Efficiency
\item Unified Buffer Cache
\item Recovery
\item Log Structured File System
\end{itemize}
\end{frame}

% XXXXXXXXXXXXXXXXXXXXXXXXXXXXXXXXXXXXXXXXXXXXXXXXXXXXXXXXXXXXXXXXXXXXXXXXXX
\section{File System Internals}
\begin{frame}[fragile]
\frametitle{File Systems Internals}
\begin{itemize}
\item File Systems
\item File-System Mounting
\item Partitions and Mounting
\item File Sharing
\item Virtual File Systems
\item Remote File Systems
\item Consistency Semantics
\item NFS
\end{itemize}
\end{frame}

% XXXXXXXXXXXXXXXXXXXXXXXXXXXXXXXXXXXXXXXXXXXXXXXXXXXXXXXXXXXXXXXXXXXXXXXXXX

\end{document}

