%%%%%%%%%%%%%%%%%%%%%%%%%%%%%%%%%%%%%%%%%%%%%%%%%%%%%%%%%%%%%%%%%%%%%%%%%
% REV154 Thu Aug 23 11:22:02 WIB 2018
% START0 Thu Jul 26 20:01:45 WIB 2018
%%%%%%%%%%%%%%%%%%%%%%%%%%%%%%%%%%%%%%%%%%%%%%%%%%%%%%%%%%%%%%%%%%%%%%%%%

\section{Week 05}
\begin{frame}[fragile]
\frametitle{Week 05 Virtual Memory:
Topics\footnote{Source: ACM IEEE CS Curricula 2013}}

\begin{itemize}
\item Review of physical memory and memory management hardware 
\item Virtual Memory 
\item Caching 
\item Memory Allocation 
\item Memory Performance 
\item Working sets and thrashing 
\end{itemize}
\end{frame}

\begin{frame}[fragile]
\frametitle{Week 05 Virtual Memory:
Learning Outcomes\footnote{Source: ACM IEEE CS Curricula 2013}}
\begin{itemize}
\item Explain memory hierarchy and cost-performance trade-offs. [Familiarity]
\item Summarize the principles of virtual memory as applied to caching and paging. [Familiarity] 
\item Describe the reason for and use of cache memory (performance and proximity, different dimension of how caches complicate isolation and VM abstraction). [Familiarity] 
\item Defend the different ways of allocating memory to tasks, citing the relative merits of each. [Assessment] 
\item Evaluate the trade-offs in terms of memory size (main memory, cache memory, auxiliary memory) and processor speed. [Assessment] 
\item Discuss the concept of thrashing, both in terms of the reasons it occurs and the techniques used to recognize and manage the problem. [Familiarity] 
\end{itemize}

\end{frame}

